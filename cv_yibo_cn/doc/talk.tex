\begin{rSection}{特邀报告}


\textbf{国际会议特邀报告}
        
\begin{description}[font=\normalfont]
%{{{

\item[{[11]}]{
``Analyzing Timing in Shorter Time: A Journey through Heterogeneous Parallelism for Static Timing Analysis'', in IEEE International Conference on Solid-State and Integrated Circuit Technology (ICSICT), Zhuhai, China, Oct 22-25, 2024. 
}

\item[{[10]}]{
``Agile Analog IC Design: from Layout Automation to Circuit Synthesis'', in IEEE International Symposium on Radio-Frequency Integration Technology (RFIT), Chengdu, China, Aug 28-30, 2024. 
}

\item[{[9]}]{
``Workshop Talk: Empowering Physical Design of VLSI Circuits with Deep Learning: from Modeling to Optimization'', in International Symposium on Computer Architecture (ISCA), Buenos Aires, Argentina, Jun 29-Jul 3, 2024. 
}

\item[{[8]}]{
``Accelerating Routability and Timing Optimization with Open-Source AI4EDA Dataset CircuitNet and Heterogeneous Platforms'', in ACM/IEEE International Conference on Computer-Aided Design (ICCAD), San Francisco, Oct 29-Nov 2, 2023. 
}

\item[{[7]}]{
``Tutorial: Deep Learning Enabled Timing Optimization in Physical Design'', in ACM/IEEE Design Automation Conference (DAC), San Francisco, Jul 9-13, 2023. 
}

\item[{[6]}]{
``Timing Analysis and Optimization on Heterogeneous CPU-GPU Platforms'', in International Workshop on Logic \& Synthesis (IWLS), Virtual, Jul 18-21, 2022. 
}

\item[{[5]}]{
``DREAMPlace: Deep Learning Toolkit-Enabled GPU Acceleration for Modern VLSI Placement'', in ACM/IEEE Design Automation WebiNar (DAWN), Virtual, Apr 11-12, 2022. 
}

\item[{[4]}]{
``DREAMPlace 3.X: Exploring Advanced Constraints and Multi-GPU Acceleration'', in China Semiconductor Technology International Conference (CSTIC), Shanghai, China, Mar 14-15, 2021. 
}

\item[{[3]}]{
``Deep Learning for Mask Synthesis and Verification: A Survey'', in IEEE/ACM Asia and South Pacific Design Automation Conference (ASPDAC), Tokyo, Japan, Jan 18-21, 2021.
}

\item[{[2]}]{
``Tutorial: GPU Acceleration in VLSI Back-end Design: Overview and Case Studies'', in IEEE/ACM International Conference on Computer-Aided Design (ICCAD), Virtual, Nov 2-5, 2020. 
}

\item[{[1]}]{
``DREAMPlace 2.0: Open-Source GPU-Accelerated Global and Detailed Placement for Large-Scale VLSI Designs'', in China Semiconductor Technology International Conference (CSTIC), Shanghai, China, Jun 26, 2020. 
}

%}}}
\end{description}

\textbf{国内会议特邀报告}
        
\begin{description}[font=\normalfont]
%{{{

\item[{[12]}]{
``深度学习赋能集成电路物理设计自动化:从建模到优化方法'', 中国计算机大会(CNCC), 横店, Oct 24-27, 2024. 
}

\item[{[11]}]{
``Agile Analog IC Design: from Layout Automation to Circuit Synthesis'', 华为模拟设计自动化技术峰会, Aug 19, 2024. 
}

\item[{[10]}]{
``AI赋能集成电路物理设计自动化:数据集,建模和优化方法'', 全国超导薄膜和超导电子器件学术研讨会, 贵阳, Aug 13-17, 2024. 
}

\item[{[9]}]{
``物理设计中的异构并行加速问题:从图理论到数值计算'', 中国计算机学会芯片大会(CCF Chip), 上海, Jul 19-21, 2024. 
}

\item[{[8]}]{
``异构计算和人工智能加速物理设计与优化'', 中国计算机学会集成电路设计与自动化学术会议(CCF-DAC), 北京, Oct 13-16, 2023. 
}

\item[{[7]}]{
``A Timing Engine Inspired Graph Neural Network Model for Pre-Routing Slack Prediction'', CCF Chip芯片大会, 南京, Jul 29-31, 2022. 
}

\item[{[6]}]{
``Exploring AI-assisted Optimization Opportunities in Placement and Routing'', 华为Strategy and Technology Workshop (STW), 深圳, Oct 14-16, 2021. 
}

\item[{[5]}]{
``A Provably Good and Practically Efficient Algorithm for Common Path Pessimism Removal in Static Timing Analysis'', ChinaDA, 北京, Jul 10-11, 2021. 
}

\item[{[4]}]{
``深度学习辅助布局布线优化'', 中国计算机学会青年精英大会 (CCF-YEF), 沈阳, May 15, 2021. 
}

\item[{[3]}]{
``DREAMPlace 3.0: Multi-Electrostatics Based Robust VLSI Placement with Region Constraints'', 东湖论坛, 武汉, Nov 28, 2020.
}

\item[{[2]}]{
``先进工艺下AI辅助芯片后端设计与制造'', 中国计算机学会集成电路设计与自动化学术会议 (CCF-DAC), 线上, Aug 10-11, 2020. 
}

\item[{[1]}]{
``基于机器学习的集成电路后端设计及加速'', 华为海思与高校技术论坛, 北京, Nov 28, 2019. 
}

%}}}
\end{description}

\textbf{国内外机构邀请报告}
        
\begin{description}[font=\normalfont]
%{{{

\item[{[13]}]{
``AI-Empowered Heterogeneous Computing for Physical Design Automation towards Timing Closure'', KAIST, Korea, Oct 21, 2024. 
}

\item[{[12]}]{
``AI-Empowered Heterogeneous Computing for Physical Design Automation towards Timing Closure'', Seoul National University, Korea, Sep 27, 2024. 
}

\item[{[11]}]{
``The Art of Formulation and Optimization in VLSI Placement for Diverse Design Scenarios'', Google DeepMind, Mountain View, California, Jul 29, 2024. 
}

\item[{[10]}]{
``AI-Empowered Heterogeneous Computing for Physical Design Automation'', Georgia Institute of Technology, Atlanta, Georgia, Jul 3, 2024. 
}

\item[{[9]}]{
``Deep Learning for Physical Design Automation of VLSI Circuits: Modeling, Optimization, and Datasets'', Synopsys, Armenia, Feb 5, 2024. 
}

\item[{[8]}]{
``Accelerating Timing Closure of IC Design with Heterogeneous Computing and Machine Intelligence'', University of Wisconsin, Madison, Nov 3, 2023. 
}

\item[{[7]}]{
``Accelerating Timing Closure of Integrated Circuits with Heterogeneous Computing and Machine Intelligence '', Arizona State University, Oct 27, 2023. 
}

\item[{[6]}]{
``Heterogenous Timing Analysis, Prediction, and Optimization of Integrated Circuits with Machine Intelligence '', National University of Singapore, Aug 29, 2023. 
}

\item[{[5]}]{
``Deep Learning for Backend Design Automation of VLSI Circuits: Modeling, Optimization, and Datasets'', Hong Kong University of Science and Technology, Apr 14, 2023. 
}

\item[{[4]}]{
``Deep Learning for Physical Design Automation of VLSI Circuits: Modeling, Optimization, and Datasets'', Chinese University of Hong Kong, Mar 23, 2023. 
}

\item[{[3]}]{
``Timing Analysis and Optimization on Heterogeneous CPU-GPU Platforms'', Synopsys, Armenia, Jan 30, 2023. 
}

\item[{[2]}]{
``Accelerating VLSI Physical Design with Parallel and Heterogeneous Computing'', Synopsys, Armenia, Jan 24, 2022. 
}

\item[{[1]}]{
``Machine Learning Based IC Backend Design and Acceleration'', Synopsys, Armenia, Apr 8, 2021. 
}

%}}}
\end{description}

\end{rSection}
